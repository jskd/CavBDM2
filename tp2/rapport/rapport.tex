\documentclass[a4paper]{article}

\usepackage[english]{babel}
\usepackage[utf8]{inputenc}
\usepackage{amsmath}
\usepackage{graphicx}
\usepackage[colorinlistoftodos]{todonotes}

\title{Concepts Avancés de Bases de données}

\author{Joaquim LEFRANC et Jérôme Skoda}

\date{\today}

\begin{document}
\maketitle

\begin{abstract}
Comment avons nous pu écrire autant de chose sur seulement deux boucle imbriqué et une fonction de merge?
\end{abstract}

\section{Comment compiler le projet}
\label{sec:introduction}

\subsection{Sans ligne de commande avec un IDE ki fé tou}

Tout d'abord, ce projet ne fonctionnera pas avec code:block. 
Nous avons passé une longue heure à configurer eclipse et nous en avons assez!



\section{Theory 2-3 pages}
\label{sec:theory}

\subsection{Two-dimensional Electron Gas}
Here, explain the concept of a 2-DEG in GaAs/AlGaAs. What is a 2-DEG and why does it arise?

\subsection{Hall Effect}
Explain the classical Hall effect in your own words. What do I measure at $B=0$? And what happens if $B>0$? Which effect gives rise to the voltage drop in the vertical direction?

\subsection{Quantum Hall Effect}
Explain the IQHE in your own words. What does the density of states look like in a 2-DEG when $B=0$? What are Landau levels and how do they arise? What are edge states? What does the electron transport look like when you change the magnetic field? What do you expect to measure?

\section{Experiment 1-2 pages}
\subsection{Fabrication}
Explain a step-by-step recipe for fabrication here. How long did you etch and why? What is an Ohmic contact?
\subsection{Experimental set-up}
Explain the experimental set-up here. Use a schematic picture (make it yourself in photoshop, paint, ...) to show how the components are connected. Briefly explain how a lock-in amplifier works.

\section{Results and interpretation 2-3 pages}
Show a graph of the longitudinal resistivity ($\rho_{xx}$) and Hall resistivity ($\rho_{xy}$) versus magnetic field, extracted from the raw data shown in figure \ref{fig:data}. You will have the link to the data in your absalon messages, if not e-mail Guen (guen@nbi.dk). Explain how you calculated these values, and refer to the theory.



\subsection{Classical regime}
Calculate the sheet electron density $n_{s}$ and electron mobility $\mu$ from the data in the low-field regime, and refer to the theory in section \ref{sec:theory}. Explain how you retrieved the values from the data (did you use a linear fit?).
Round values off to 1 or 2 significant digits: 8.1643 ~= 8.2. Also, 5e-6 is easier to read than 0.000005.

!OBS: This part is optional (only if you have time left).
Calculate the uncertainty as follows: \newline $u(f(x, y, z)) = \sqrt{(\frac{\delta f}{\delta{x}} u(x))^{2} + (\frac{\delta f}{\delta{y}} u(y))^{2} + (\frac{\delta f}{\delta{z}} u(z))^{2}}$, where $f$ is the calculated value ($n_{s}$ or $\mu$), $x, y, z$ are the variables taken from the measurement and $u(x)$ is the uncertainty in x (and so on).

\subsection{Quantum regime}
Calculate $n_{s}$ for the high-field regime.
Show a graph of the longitudinal conductivity ($\rho_{xx}$) and Hall conductivity($\rho_{xy}$) \textbf{in units of the resistance quantum} ($\frac{h}{e^{2}}$), depicting the integer filling factors for each plateau.
Show a graph of the plateau number versus its corresponding value of $1/B$. From this you can determine the slope, which you use to calculate the electron density.
Again, calculate the uncertainty for your obtained values.

\section{Discussion 1/2-1 page}
Discuss your results. Compare the two values of $n_{s}$ that you've found in the previous section. Compare your results with literature and comment on the difference. If you didn't know the value of the resistance quantum, would you be able to deduce it from your measurements? If yes/no, why?

\newpage
\section{Some LaTeX tips}
\label{sec:latex}
\subsection{How to Include Figures}



\end{document}
              
